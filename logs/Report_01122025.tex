\documentclass[11pt,a4paper]{article}

\usepackage{amsmath,amssymb,graphicx,float,hyperref,caption}
\usepackage{geometry}
\geometry{margin=1in}

\title{\textbf{Multi-UAV Path Planning, Control, and Elastic Band Optimization}\\
	{\large System Architecture, Theory, and Development Report}}
\author{Diyari + Ilyas }
\date{\today}

\begin{document}
	\maketitle
	
	\begin{abstract}
		This report presents the development of a full-stack multi-UAV navigation framework.
		The system integrates global planning (A*, RRT), reactive local planning (Potential Fields), 
		path smoothing using Elastic Band deformation, dynamic obstacles, and multi-agent interactions.
		Unlike holonomic simplifications found in literature, our UAVs are simulated using 
		a \emph{non-holonomic bicycle model} with steering dynamics, matched exactly to our implementation.
		The report documents all major modules, the theoretical background that supports them, the engineering
		decisions made during development, and the iterative improvements that led to a stable working system.
	\end{abstract}
	
	\section{Introduction}
	The objective of this project is to design, implement, and analyze a full multi-UAV navigation pipeline.
	The system includes:
	\begin{itemize}
		\item Global path planners (A* \cite{astar}, RRT \cite{rrt})
		\item Local reactive potential field controller
		\item Elastic Band path deformation for smoothing
		\item A realistic non-holonomic UAV model based on bicycle dynamics
		\item Multi-agent interaction forces
		\item Dynamic obstacle avoidance
	\end{itemize}
	
	The entire implementation is custom-built in Python. 
	This report combines theory with code-based descriptions of our system.
	
	\section{Environment and Occupancy Grid}
	The environment is represented using an occupancy grid defined in \texttt{environment.py} :contentReference[oaicite:0]{index=0}.
	The grid stores static obstacles, inflated obstacles for planning, and dynamic obstacles separately.
	
	Grid conversion:
	\[
	g_x = \left\lfloor \frac{x}{res} \right\rfloor,\qquad
	g_y = \left\lfloor \frac{y}{res} \right\rfloor.
	\]
	
	Inflated obstacles are generated using binary dilation:
	\[
	O_{\text{inflated}} = O \oplus B_r.
	\]
	
	Dynamic obstacles are integrated but not drawn in the base map to prevent visual clutter.
	
	\section{Non-Holonomic UAV Model}
	Our UAV is not holonomic. 
	It uses the bicycle model implemented in \texttt{uav\_controller.py} :contentReference[oaicite:1]{index=1}.
	
	\subsection{State Vector}
	\[
	s = (x, y, \theta, v, \phi)
	\]
	
	\subsection{Continuous-Time Kinematics}
	\[
	\dot{x} = v\cos\theta,\qquad
	\dot{y} = v\sin\theta,
	\]
	\[
	\dot{\theta} = \frac{v}{L}\tan\phi,
	\]
	\[
	\dot{v} = a,\qquad \dot{\phi} = \omega.
	\]
	
	This model captures steering constraints and non-holonomic motion, unlike simple point-mass approximations.
	
	\subsection{Control Logic (Simplified Explanation)}
	The potential field computes a desired force vector:
	\[
	\mathbf{f} = (f_x, f_y).
	\]
	
	Controller steps:
	\begin{enumerate}
		\item Compute desired heading:
		\[
		\theta_d = \arctan2(f_y, f_x)
		\]
		\item Heading error:
		\[
		e_\theta = \theta_d - \theta
		\]
		\item Convert to steering angle command:
		\[
		\phi_d = \arctan\left(\frac{2L\sin(e_\theta)}{v + \epsilon}\right)
		\]
		\item Compute acceleration based on force magnitude.
	\end{enumerate}
	
	This yields realistic non-holonomic behavior.
	
	\section{Simulation Logic}
	The main simulation loop is implemented in \texttt{simulation.py} :contentReference[oaicite:2]{index=2}.
	
	Key logic includes:
	\begin{itemize}
		\item Dynamic obstacle updates
		\item Event-based replanning (if waypoint becomes occupied)
		\item Pure pursuit for global planners
		\item Potential-field-based control for reactive planners
		\item UAV--UAV repulsion forces
	\end{itemize}
	
	Two UAVs operate simultaneously with independent goals.
	
	\section{A* Global Planner}
	Implemented in \texttt{a\_star.py} :contentReference[oaicite:3]{index=3}.
	
	A* minimizes
	\[
	f(n)=g(n)+h(n)
	\]
	using Chebyshev heuristic for 8-connected grids:
	\[
	h(n)=\sqrt{2}\min(dx,dy) + |dx-dy|.
	\]
	
	Path reconstruction is performed using parent pointers and converted back to world coordinates.
	
	\section{RRT Planner}
	Implemented in \texttt{rrt.py} :contentReference[oaicite:4]{index=4}.
	
	Key components:
	\begin{itemize}
		\item Random sampling
		\item Nearest-node search
		\item Steering toward sample:
		\[
		x_{new}=x_{near}+\frac{v}{\|v\|}\cdot step
		\]
		\item Collision checking
		\item Path smoothing with collision-safe shortcutting
	\end{itemize}
	
	The planner includes a limit on maximum iterations for safety.
	
	\section{Potential Field Local Planner}
	Defined in \texttt{potential\_field.py} :contentReference[oaicite:5]{index=5}.
	
	\subsection{Attractive Potential}
	\[
	\mathbf{F}_{att} = k_{att}\frac{p_{goal}-p}{\|p_{goal}-p\|}
	\]
	
	\subsection{Obstacle Repulsion}
	For an obstacle within influence radius $d_0$:
	\[
	\mathbf{F}_{obs} = k_{rep}
	\left(\frac{1}{d}-\frac{1}{d_0}\right)\frac{1}{d^2}\frac{p - p_o}{\|p - p_o\|}.
	\]
	
	\subsection{Inter-UAV Repulsion}
	\[
	\mathbf{F}_{uav} = k_{rep,uav}
	\left(\frac{1}{d}-\frac{1}{d_{uav}}\right)\frac{1}{d^2}(p_i - p_j).
	\]
	
	\section{Elastic Band Path Optimization}
	Implemented in \texttt{elasticBand.py} :contentReference[oaicite:6]{index=6}.
	
	The path is represented as a series of bubbles with radii based on clearance.
	
	\subsection{Contraction Force}
	\[
	\mathbf{F}_{cont} = k_c 
	\left(
	\frac{b_{i-1}-b_i}{\|b_{i-1}-b_i\|} +
	\frac{b_{i+1}-b_i}{\|b_{i+1}-b_i\|}
	\right)
	\]
	
	\subsection{Repulsive Force}
	Based on potential field weight:
	\[
	\rho(b)=k\left(\frac{1}{r(b)}-\frac{1}{d_{safe}}\right)
	\]
	
	A numerical gradient is used to compute repulsion direction.
	
	\subsection{Bubble Insertion \& Removal}
	If two bubbles do not overlap, a new bubble is inserted.  
	If two neighbors overlap even without $b_i$, the bubble is removed.
	
	\section{Formation Control Experiments}
	The preliminary formation experiments in \texttt{formation\_controller.py} :contentReference[oaicite:7]{index=7}
	were based on:
	\begin{itemize}
		\item Bearing-based formation graphs
		\item Scaling of formations
		\item Distance-maintenance proportional control
	\end{itemize}
	
	Although non-holonomic constraints caused instability, the experiments provided valuable insights.
	
	\section{Conclusion}
	This project successfully integrates multiple planning and control techniques into a coherent multi-UAV navigation framework.
	The system demonstrates:
	\begin{itemize}
		\item Full non-holonomic control
		\item Global \& local planning integration
		\item Elastic Band smoothing
		\item Dynamic obstacle handling
		\item Multi-agent interaction
	\end{itemize}
	
	It provides a strong foundation for future extensions such as cooperative planning or MPC-based control.
	
	\bibliographystyle{plain}
	\begin{thebibliography}{9}
		\bibitem{astar} Hart, P.E., Nilsson, N.J., Raphael, B. (1968). A Formal Basis for the Heuristic Determination of Minimum Cost Paths.
		\bibitem{rrt} LaValle, S.M. (1998). Rapidly-exploring random trees: A new tool for path planning.
	\end{thebibliography}
	
\end{document}
